\documentclass[a4paper,12pt,openany]{report}

\usepackage{titling}
\usepackage[margin=0.5in,headsep=0.25in,footskip=0.25in]{geometry}
\usepackage{graphicx}
\usepackage{float}


\newcommand{\subtitle}[1]{
	\posttitle{
		\par\end{center}
		\begin{center}\large#1\end{center}
		\vskip0.5em
	}
}

\setlength{\parindent}{12pt}


% document
\begin{document}

% Title Page
\title{CITS3001 Algorithms and Artificial Intelligence}
\subtitle{Threes Artificial Intelligence Project}
\author{Mitchell Pomery (21130887)\\
Kieran Hannigan (21151118)}
\maketitle
\clearpage

% Introduction
\section*{The Task}

\paragraph \indent
We were tasked with the creation of an artificial intelligence to play the game Threes\cite{threes}. Threes is a simple mobile game where you are given a four by four grid occupied by tiles. The aim is to merge tiles until the board is full, and no more merges are possible. You are only able to merge a one tile with a two tile, and any other tile with one of the same number.

\paragraph \indent
The specifications for the artificial intelligence stated the following:

\begin{itemize}
	\item All upcoming tiles are known
	\item The location of the placed tile is not random
	\item The AI needs to make 5 to 10 moves per second
\end{itemize}

% Language Choice
\paragraph \indent
Python was chosen as an initial language for the artificial intelligence due to the ability to rapidly prototype different algorithms.
The intention was to later port the final algorithm to C to see an increase in performance, and henceforth let the bot look further into the future for better moves.

% Algorithms
\section*{Algorithms}

\subsection*{Making Moves}
\paragraph \indent
There are two ways the moves can be implimented in the bot, either each possible move seperately, or by rotating the board and treating all moves as the one direction.

\subsection*{Naive}
\paragraph \indent
Initially an naive algorithm was implimented to play Threes one move at a time.
It would look at the four possible moves, left, right, up and down, and take the one that gives it the highest score.
The naive algorithm is useful as a heuristic later in the A* implimentation

\subsection*{Minimax}
\paragraph \indent
A minimax implimentation was investigated due to the similarities between Threes and 2048, and the already existing 2048 AI by Matt Overlan\cite{2048ai}.
In 2048 the next tile is places randomly, while the specifications state that the next tile will be placed in the lexographically lowest location.
This difference means that the Threes bot does not end up playing against anything, and so minimax is not a relevant algorithm.

\subsection*{A Star}
\paragraph \indent

% bibliography
\begin{thebibliography}{20} 
 \bibitem{threes} THREES - A tiny puzzle that grows on you. 2014. THREES - A tiny puzzle that grows on you. [ONLINE] Available at: http://asherv.com/threes/. [Accessed 28 May 2014].
 \bibitem{2048ai} 2048. 2014. 2048. [ONLINE] Available at: http://ov3y.github.io/2048-AI/. [Accessed 28 May 2014].
\end{thebibliography} 


\end{document}